\documentclass[letter,12pt]{article}
\usepackage[utf8]{inputenc}
\usepackage[T1]{fontenc}
\usepackage[spanish,es-tabla]{babel}
\usepackage{amsmath,amssymb,amsfonts}
\def\beq{\begin{equation}}
\def\eeq{\end{equation}}
\newcommand{\derivadaN}[3]{\frac{d^{#3}#1}{d#2^{#3}}}
\def\derivada#1#2{\frac{d#1}{d#2}}
\title{Sobre la Importancia del Agua en la Navegación Marítima}
\author{Rogelio Rojas Izquierdo}
\date{30 de Febrero de 2018}

\begin{document}
\maketitle
\begin{abstract}
En este trabajo resaltamos la importancia
 del agua en la navegación marítima. Concluimos
  que no es posible navegar sin agua a no ser que 
sea
 por Internet.
\end{abstract}
\section{Introducción}
Creemos firmemente que la navegación marítima 
se fundamenta en el principio de Arquímedes~\cite{arquimedes}
 sobre cuerpos flotantes en líquidos como es el agua.
\subsection{Principio de Arquímedes}
Todo cuerpo sumergido en un \textit{fluido} 
experimenta una \textbf{fuerza de empuje} hacia
 arriba, igual al peso del volumen del fluido
  desalojado.

\section{Fundamentos Físicos}

Según Albert Einstein $E=m c^{2}$. Esta
 fórmula le dio su fama en todos los niveles. Según
  Einstein, la masa de un cuerpo depende de su velocidad
  , de modo tal que,
\begin{equation}
m=\frac{m_{0}}{\sqrt{1-\frac{v^{2}}{c^{2}}}},
\label{masa}
\end{equation}
donde $m_0$ es la masa
 en reposo del cuerpo y $c$ la velocidad
  de la luz en el vacío. Según (\ref{masa}) no es posible
   que un cuerpo supere la velocidad de la luz.

De estas
 ecuaciones se deduce que,
$$E=\frac{m_0 c^2}{\sqrt{1-\frac{v^2}{c^2}}}.$$
Las transformaciones de Lorentz se pueden escribir como:
\begin{eqnarray}
t^{\prime}&=&\gamma\left( t +\frac{v}{c^2}x\right)\label{t-lorentz}\\
x^{\prime}&=&\gamma\left(x + v t\right)\label{x-lorentz}
\end{eqnarray}
donde $\gamma = \left[1-\frac{v^2}{c^2}\right]^{-\frac{1}{2}}$. Este sistema de transformaciones se pueden escribir como una única ecuación matricial:  
$$\left(
\begin{array}{c}
ct^{\prime}\\
x^{\prime}
\end{array}
\right)
=\left(
\begin{array}{c c}
\gamma & \gamma\beta  \\
\gamma\beta & \gamma
\end{array}
\right)
\left(
\begin{array}{c}
ct\\
x
\end{array}
\right)
$$

Ahora vamos a usar los paquetes de AMS. La matriz,
$$\Gamma=\left(
\begin{matrix}
\gamma&\gamma\beta\\
\gamma\beta&\gamma
\end{matrix}
\right)
$$
donde $\gamma$ y $\beta$ pertenecen a $\mathbb{R}$, 
son los coeficientes relativistas dependientes de la velocidad.

\section{Listas}
Ahora una lista:

\begin{enumerate}
 \item Hola
 \begin{enumerate}
 \renewcommand{\labelenumii}{\labelenumi.\roman{enumii}}
 \item Buenos días.
 \item Buenas tardes.
 \end{enumerate}
 \item Adiós
 \end{enumerate}
 
Otra lista:
\begin{enumerate}
\item Álgebra.
	\begin{enumerate}
	\item grupos y Anillos.
	\item Cuerpos.
	\item Espacios Vectoriales.
	\end{enumerate}
\item Análisis.
	\begin{enumerate}
	\item Análisis real.
	\item Análisis complejo.
	\item Análisis funcional.
	\end{enumerate}
\item Geometría.
	\begin{itemize}
	\item Geometría Euclídea.
		\begin{itemize}
		\renewcommand{\labelitemii}{$\partial$}
		\item En el plano.
		\item En el Espacio.
		\end{itemize}
	\item  Geometría rimaniana.
		\begin{itemize}
		\item Geometría extrínseca.
		\item Geometría intrínseca.
		\end{itemize}
	\end{itemize}
\item Lógica.
	\begin{enumerate}
	\item Teoría de conjuntos.
	\item Teoría de modelos.
	\end{enumerate}
\end{enumerate}
Y finalmente una lista de descripción:
\begin{description}
\item[España:] Es un país que está 
al sur de Europa.
\item[Colombia:] Es un país que se encuentra en América y
tiene costa del Caribe.
\end{description}

\section{Macros}
Ahora un desarrollo del señor   ``sastre'':
\beq
f(x+h)=f(x) + h\derivada{f(x)}{x} + \frac{h^2}{2!}\derivadaN{f(x)}{x}{2} + \dots
\eeq

\section{Tablas}
Tabla sencilla:
\begin{table}[h]
   \centering
   \begin{tabular}{|c||c|l|}
   \hline
   País & Capital & Idioma\\
   \hline
   \hline
   Francia & París & francés\\
   \hline
   España & Madrid & español\\
   \hline
   Italia & Roma & italiano\\
   \hline
   Colombia & Bogotá & español\\
   \hline
   \end{tabular}
   \caption{Lista de países con sus capitales}
   \label{paises}
\end{table}
En la tabla \ref{paises} vemos varios 
países con sus respectivas capitales e idioma más hablado.
\section{Figuras}
 Según Pitágoras~\cite{pitagoras}, el cuadrado de la {\it hipotenusa} 
 es igual a la suma de los cuadrados de los
 {\it catetos} en un triángulo rectángulo:
 $$h^2= a^2 +b^2,$$
como podemos comprobar en la  figura  \ref{triangulo}.
\begin{figure}
\centering
\begin{picture}(300,100)
\put(20,20){\line(1,0){150}}\put(20,20){\line(2,1){150}}
\put(170,20){\line(0,1){75}}
\put(50,24){$\theta$}\put(85,60){$a$}
\put(85,5){$\sqrt{a^2-b^2}$}\put(180,50){$b$}
\put(160,20){\line(0,1){10}}\put(160,30){\line(1,0){10}}
\put(48,20){\line(-1,4){3}}
\put(220,70){$\displaystyle\mathrm{sen}\theta=\frac ba$}
\put(220,30){$\displaystyle\cos\theta=\frac{\sqrt{a^2-b^2}}{a}$}
\end{picture}
\caption{ Un triángulo rectángulo.}
\label{triangulo}
\end{figure}   
\section{Conclusiones}
 Ahora vemos de forma clara y
  nítida la importancia del agua
   en la navegación marítima.

\begin{thebibliography}{99}
\bibitem{arquimedes} Arquímedes. {\it  Mi Principio}.
 Phys.Lett.{\bf B}73 (1980).
\bibitem{pitagoras} Pitagoras. {\it Triángulos}.
 Phys.Rev.{\bf D}23 (1985).
\end{thebibliography}
\end{document}
